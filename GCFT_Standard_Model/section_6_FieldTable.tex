\section{The Full GCFT Coherence Field Table}

This table replaces the Standard Model particle classification with a resonance-based view rooted in $\Xi$-field topology and phase dynamics. Each entry corresponds to a distinct standing field configuration — not a particle in space, but a coherent structure in phase.

\begin{table}[htbp]
\footnotesize  % Smaller font to fit content
\setlength{\tabcolsep}{6pt}  % Reduce horizontal padding
\renewcommand{\arraystretch}{1.2}  % Slightly increase vertical spacing
\centering
\caption{Coherence Field Table: Resonant GCFT Structures}
\label{tab:coherence_field_table}
\begin{tabularx}{\textwidth}{l >{\centering\arraybackslash}X >{\centering\arraybackslash}X >{\centering\arraybackslash}c >{\centering\arraybackslash}c}
\toprule
\textbf{Name} & \textbf{Core} & \textbf{Phase} & \textbf{Charge} & \textbf{Field Symbol} \\
\midrule
\csvreader[
    late after line=\\,
    head to column names
]{coherence_table.csv}{}%
{\Name & \Core & \Phase & \Charge & \FieldSymbol}
\bottomrule
\end{tabularx}
\end{table}

The ``mass threshold'' indicates the coherence energy needed to maintain that structure. Stability follows from the field’s ability to preserve synchrony under perturbation, and decay pathways emerge from topological unwinding or phase reconfiguration.

This table serves as the reference map for interpreting all particle-like phenomena in GCFT: not as matter points or mediators, but as snapshots of the field’s capacity to hold tension, memory, and synchrony.
\clearpage

\subsection {Ontological Summary Table}
\label{sec:ontology_summary}

To clarify the conceptual shift introduced by General Coherence Field Theory, the following table compares foundational concepts from the Standard Model and General Relativity with their reinterpretation in GCFT.

\begin{table}[H]
\centering
\begin{tabular}{lll}
\toprule
\textbf{Concept} & \textbf{Standard Model / GR} & \textbf{GCFT Interpretation} \\
\midrule
Particle        & Point-like excitation               & Standing $\Xi$-resonance (knot) \\
Force           & Mediated by bosons                  & Coherence rupture or gradient tension \\
Mass            & Higgs mechanism or curvature        & Luxion compression threshold \\
Charge          & U(1) gauge symmetry                 & Topological winding in $\arg(\Xi)$ \\
Spin            & Quantum number                      & Phase vortex handedness \\
Photon          & U(1) gauge boson                    & Free luxion ripple (stable) \\
Gluon           & SU(3) exchange boson                & Collective torsion harmonizer \\
Gravity         & Curved spacetime metric             & $\Xi$-compression gradient: $-\nabla \Phi_\Xi$ \\
Time            & External dimension                  & Phase drift and decoherence slope \\
Vacuum          & Empty background                    & Low-tension coherent $\Xi$ substrate \\
Wavefunction    & Abstract quantum amplitude          & Real field phase configuration \\
Collapse        & Measurement-triggered jump          & Decoherence beyond resonance margin \\
\bottomrule
\end{tabular}
\caption{Ontological comparison between conventional physics and GCFT. In GCFT, all physical behavior emerges from dynamic phase structure in the continuous coherence field~$\Xi$.}
\label{tab:gcft_ontology_table}
\end{table}
